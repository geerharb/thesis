\section{Conclusion}
SAFT-VR uses perturbation to incorporate interactions up to some unknown length scale L. GRG builds upon SAFT-VR by incorporating even longer wavelengths in an iterative manner. Unfortunately the unknown length scale is a free parameter. Free parameters are usually varied to find the best fit with experiment, but extra free parameters that aren't needed usually saps a model of its predictive abilities. In the extreme case, a bunch of experiments can be interpolated to 'predict' what will happen in the valid range of the data, yet the interpolation will fail far from the actual data. This failure to make predictions can also happen to models with too many free parameters which is undesirable for everyone involved.

This project explored how to fix the arbitrary parameter L in GRG by instead comparing the results of Monte Carlo simulations to SAFT-VR. Given that SAFT-VR only incorporates interactions up to some length scale L, the Monte Carlo simulations with a box size of length L should be able to reproduce the results of SAFT-VR. This project did vary the box size of the Monte Carlo simulations and found a 'best' fit length of roughly L=(7.6). To test this best fit, the phase coexistence was found for three box sizes and compared to SAFT-VR. Boxes much smaller than 7.6 had a critical point higher than SAFT-VR, while box sizes much larger than 7.6 had a critical point below SAFT-VR.

While working on this project, I honestly thought the noise of the simulations would drown out the coexistence points. It was interesting to see the coexistence plots actually see a signal as low as (0.00006) (see fig~\ref{fig:FdispVsff}). Work for the future would include running multiple simulations to increase the signal to noise ratio, auto detection of outliers, and an squeue interface that can fairly balance the workload between multiple users.