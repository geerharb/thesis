\section{Introduction}

Outline:
Introduction to liquid vapor coexistence
Demonstrate the problem
Fix the problem
arbitrary L+outline rest of paper

The liquid-vapor phase coexistence can be found by just depositing some liquid into a sealed container. Some of the liquid will evaporate into the jar creating a vapor pressure. Provided there is enough liquid to create the necessary vapor pressure, the system will now contain both a liquid and vapor at the same time. A typical thermodynamics problem will first find the free energy of the liquid by itself, then find the free energy of the vapor by itself, and finally minimize the total free energy while keeping the total number of atoms between the liquid and solid at a constant N. By keeping the atoms in a sealed box we are enforcing the $Pressure_{liquid}=Pressure_{vapor}$, and by allowing the atoms to move between liquid and vapor state we are enforcing that $Mu_{liquid}=Mu_{vapor}$.

The typical thermodynamic analysis of treating each phase as a separate entity is usually fine under normal conditions, but a problem occurs as the liquid approaches the critical point. The critical point occurs as the temperature of the liquid is raised; this causes thermal vibrations in the liquid to reduce the liquid density, and it also causes a denser vapor as the atoms are able to escape the surface of the liquid easier. Basically this causes the density of the liquid to approach the density of the gas as the system reaches the critical point.

*expand using 2d liquid and vapor graphs near critical point*

Near the critical point the liquid and vapor states start to mix, which means the typical thermodynamic analysis of treating each phase separately no longer works. One method to get around this problem is called renormalization group theory (RGT). RGT recognizes that oscillations in the density occur at all temperatures rather than just at the critical point. These oscillations should naturally have an energy associated with them, so RGT starts with the first approximation of treating a liquid/vapor as a single phase, but then applies thermodynamics to the oscillations. **ADD MORE**

Within RGT, the oscillations are found by starting at a particular box size called L. The oscillations within a box of length L can't have oscillations longer than the size L simply because of the periodic boundary conditions. The calculations usually start at a small box of length L to find the short wavelenth oscillations, then double the box size to find even longer wave length oscillations. The problem is this box length L is usually an arbitrary parameter that needs to be fit to either the experimental data or to large Monte-Carlo simulations. An alternative approach is to realize the base theory that RGT revolves around already contains the smallest scale oscillations. The alternative approach exploits this by trying to fit Monte-Carlo simulations to SAFT. To do this the box size is varied until the Monte-Carlo simulations can recreate the thermodynamics of SAFT. *add more*

\newpage

Test of fancyhdr page.
