%\section{}
%\pagestyle{plain}   %Use if you do not want the fancy headers and footers... 
					 %({plain} -> just page number, {empty} -> nothing)...
					 % this will change all pages that come after this tex file...
					 % so make sure to reset the pagestyle to {fancy} at the end of this file
Outline:
SAFT\newline
liquid-vapor+run-absolute\newline
How comparisons were made\newline
\section{Free energy seperation}
The Monte Carlo simulations can be partitioned into high and low temperature regimes. At low temperature the interactions dominate, while at high temperatures the entropy dominates. Both regimes can be partitioned yet again by seperating the kinetic energies into an ideal gas term. In terms of equations, we start with the partition function and seperate the summation into the three partitions:
$\newline Z=\frac{1}{N!}\sum_i e^{-\beta\cdot E_i}
\newline Z=\frac{1}{N!}\int\cdot\cdot\cdot\int e^{-\beta\cdot [\{P_{1}^2/{2m}+P_{2}^2/{2m}+...\}+V(\vec{R_1},\vec{R_2},...)]}d\vec{P_1}\cdot d\vec{P_2}\cdot\cdot\cdot d\vec{P_N}\cdot d\vec{R_1}\cdot d\vec{R_2}\cdot\cdot\cdot d\vec{R_N}\newline$
The momentum integrals are independent of the position integrals, so we seperate the momentum integrals which we then incorporate into an ideal gas term by multiplying by $V^N/V^N$.
$\newline Z=\boldsymbol{Z_{ideal} \cdot \frac{1}{V^N}}\cdot \int\cdot\cdot\cdot\int e^{-\beta\cdot V(\vec{R_1},\vec{R_2},...)}d\vec{R_1}\cdot d\vec{R_2}\cdot\cdot\cdot d\vec{R_N}\newline$
where $Z_{ideal}=V^N\cdot\frac{1}{N!}\int\cdot\cdot\cdot\int e^{-\beta\cdot [P_{1}^2/{2m}+P_{2}^2/{2m}+...]}d\vec{P_1}\cdot d\vec{P_2}\cdot\cdot\cdot d\vec{P_N}\newline$
The remaining potential energy term is seperated into the low temperature and high temperature regime. At high temperatures, the atom energies are much higher than the potential well; this means the finite potential well can be considered zero. However; even at infinite temperature the atoms are assumed to be a hard sphere. This hard sphere behavior is incorporated into the partition function by multiplying by $Z_{HS}/Z_{HS}$.
$\newline Z=Z_{ideal}\cdot \boldsymbol{Z_{HS}\cdot \frac{1}{Z_{HS}}}\cdot \frac{1}{V^N}\cdot \int\cdot\cdot\cdot\int e^{-\beta\cdot V(\vec{R_1},\vec{R_2},...)}d\vec{R_1}\cdot d\vec{R_2}\cdot\cdot\cdot d\vec{R_N}\newline$
where $Z_{HS}=\int\cdot\cdot\cdot\int e^{-\beta\cdot V_{HS}(\vec{R_1},\vec{R_2},...)}d\vec{R_1}\cdot d\vec{R_2}\cdot\cdot\cdot d\vec{R_N}\newline$
The remaining terms are grouped under the dispersion term.
$\newline Z=Z_{ideal}\cdot Z_{HS}\cdot \boldsymbol{Z_{disp}}\newline$
where $Z_{disp}= \frac{1}{Z_{HS}}\cdot \frac{1}{V^N}\cdot \int\cdot\cdot\cdot\int e^{-\beta\cdot V(\vec{R_1},\vec{R_2},...)}d\vec{R_1}\cdot d\vec{R_2}\cdot\cdot\cdot d\vec{R_N}\newline$

\section{Free Energy via Monte Carlo Simulations}
\section{SAFT}
SW.fid+SW.fdisp+SW.fhs Explain.

\subsection{$S_{exc.\infty}$}

\subsection{Normalized Free Energy}


%\pagestyle{fancy}   % Reset all pages after this file to fnacy headers